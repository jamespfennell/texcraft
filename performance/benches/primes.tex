% While loop: repeatedly executes #2 while \`\ifnum #1\` is true
\def\while#1#2{
  \ifnum #1
    #2%
    \while{#1}{#2}\fi
}

% Set up the sub-array of \count
\countdef \DStart 1000
\countdef \DEnd 1001
\countdef \pos 1002
\DStart = 2000
\DEnd = 2000
\def\getD#1#2{%
    \pos = #1%
    \advance \pos by \DStart%
    #2= \count \pos \relax%
}
\def\setD#1#2{%
    \pos = #1%
    \advance \pos by \DStart%
    \global \count \pos = #2\relax%
}

% Temporary variables
\countdef \i 1003
\countdef \j 1004
\countdef \k 1005
\countdef \l 1006
\countdef \m 1007
\countdef \n 1008
\countdef \o 1009

% This variable stores the next number to be considered for primality
\countdef \q 1010
\q = 2

% Inserts an element with key #1 and value #2 into the multimap
%
% Example: \insert{3}{5}
%
% Uses temporaries: none
\def\insert#1#2{
    {
        \i = \DStart\relax
        \while{\i < \DEnd}{
            % If we found an empty slot, insert the element
            \getD \i \j
            \ifnum \j = 0 \relax
                \setD{\i}{#1}
                \advance \i by 1
                \setD{\i}{#2}
                % Set \i to be 1+\DEnd so that we exit the while loop but don't insert the element at the end
                \i = \DEnd
                \advance \i by 1
            \else
                \advance \i by 2
            \fi
        }
        % If we didn't find an empty slot, insert the element at the end
        \ifnum \i = \DEnd \relax
            \setD{\i}{#1}
            \advance \i by 1
            \setD{\i}{#2}
            \global\advance\DEnd by 2
        \fi
    }
}

% Runs #1{<key>}{<value>} for each key, value pair in the multimap
%
% Example: \def\visitor#1#2{Key=\the #1, Value=\the#2.} \foreach\visitor
%
% Uses temporaries: \i, \j, \k
\def\foreach#1{
    \i = \DStart\relax
    \while{\i < \DEnd}{
        \getD \i \j
        \ifnum \j > 0 \relax
            \advance \i by 1
            \getD \i \k
            #1\j\k
            \advance \i by 1
        \else
            \advance \i by 2
        \fi
    }
}

% Prints the multimap as lines of the form D[<key>] = <value>.
%
% Uses \foreach.
\def\print{
    \def\printKeyValue##1##2{
        D[\the ##1] = \the ##2\newline
    }
    \foreach{\printKeyValue}
}

% Deletes all elements from the map whose key is #1.
%
% Example: \delete{4}
%
% Uses temporaries: none
\def\delete#1{
    {
        \i = \DStart\relax
        \while{\i < \DEnd}{
            \getD \i \j
            \ifnum \j = #1 \relax
                \setD{\i}{0}
            \fi
            \advance \i by 2
        }
    }
}

% TODO: doc \propagateD
%
% Uses \foreach and uses temporaries: \m
\def\propagateD{
    \m = 0
    \def\visitor##1##2{
        \ifnum ##1 = \q \relax
            \l = \q
            \advance \l by ##2\relax
            \insert{\l}{##2}
            \advance \m by 1
        \fi
    }
    \foreach{\visitor}
}

\def\nextPrime{
    \n = -1\relax
    \while{\n < 0}{
        \propagateD
        % if there was no element in the multimap, this is a prime
        \ifnum \m = 0 \relax
            \n = \q
            \multiply \n by \n \relax
            \insert{\n}{\q}
            \n = \q
        \else
            \delete{\q}
        \fi
        \advance \q by 1
    }
}

\while{\o < 130}{
  \nextPrime
  \advance\o by 1
}
\the \n

\end
