% Digits of pi using the Rabinowitz and Wagon spigot algorithm [1]
%
% This TeX script calculates the first n digits of pi. It uses \count registers
% for working memory and is runnable on both Texcraft and pdfTeX. The variable n 
% is specified at the end of the script, and is restricted as follows:
%
% - pdfTeX: n cannot be much larger than 1000 as pdfTeX runs out of stack space.
%
% - Texcraft: n cannot be larger than 7559 as the algorithm requires n + (10n)//3 + 12
%   integers in memory, and we assume 2^15 registers, as in pdfTeX.
%   The playground has a version of this script that uses Texcraft allocation commands
%   instead of \count registers and is thus not subject to this restriction.
%   But also, for some currently unknown value of n, 32-bit integer overflow will occur
%   and the result will be incorrect. From the original paper [1] we know that this n
%   is at least 5000. 
%
% \count register layout:
%
% - [0, 25196): used for the length 10n/3 working array
% - [25196, 32756): used for storing the n results
% - [32756, 32768=2^15): used for 12 named variables via \countdef
%
% [1] http://www.cs.williams.edu/~heeringa/classes/cs135/s15/readings/spigot.pdf

% While loop: repeatedly executes #2 while `\ifnum #1` is true
\def\while#1#2{
  \ifnum #1
    #2%
    \while{#1}{#2}\fi
}

\countdef \n 32767
\n = 1000

\countdef \i 32764
\countdef \k 1
\countdef \j 32763
\j = 0
\while{\j < \n}{
  \advance \j by 1
  %
  \i = \n
  \while{\i > 0}{
    \advance \i by -1
    \advance \k by 1
    \advance \k by 1
    \advance \k by 1
    \advance \k by 1
    \advance \k by 1
    \advance \k by 1
    \advance \k by 1
    \advance \k by 1
    \advance \k by 1
    \advance \k by 1
    \advance \k by 1
  }
}

% My Ryzen 3700 CPU is 4.6 MHz = 4600000000 Hz during single core burst workloads
% Clock cycles per nanosecond is 4.6
% cost of one advance:
% - Texcraft: 500 nanonseconds = 2300 clock cycles
% - pdfTeX: 120 nanoseconds = 550 clock cycles

\the \i

% To run on pdfTeX, uncomment the following line.
\end
